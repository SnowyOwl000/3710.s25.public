\item\problemnumber{3}{3}{14}{-}{-}
There is a more efficient algorithm (in terms of the number of multiplications
and additions used) for evaluating polynomials than the conventional algorithm
described in Exercise~3-3-13. It is called Horner's method. This pseudocode
shows how to use this method to find the value of $a_nx^n+a_{n-1}x^{n-1}+\cdots
+a_1x+a_0$ at $x=c$.
\begin{algorithm}[H]
    \caption{Horner's method}
    \begin{algorithmic}[1]
        \Procedure{Horner}{$c,a_0,a_1,\ldots,a_n$}
            \State $y\gets a_n$
            \Statex
            \For{$i\gets n-1{\bf\ downto\ }0$}
                \State $y\gets y\cdot c+a_i$
            \EndFor
            \Statex
            \State {\bf return\ }$y$
        \EndProcedure
    \end{algorithmic}
\end{algorithm}
\begin{list}{\textbf{\alph{enumii}.}}{\usecounter{enumii}}
\item Evaluate $3x^2+x+1$ at $x=2$ by working through teach step of the
    algorithm showing the values assigned at each assignment step.
\item Exactly how many multiplications and additions are used by this algorithm
    to evaluate a polynomial of degree $n$ at $x=c$? (Do not count additions
    used to increment the loop variable.)
\end{list}
\vskip12pt
\ifanswers
\textcolor{blue}{
\textbf{Answer:}\\[6pt]
\begin{list}{\textbf{\alph{enumii}.}}{\usecounter{enumii}}
\item Part a answer
\item Part b answer
\end{list}
}
\newpage
\fi
