\item\problemnumber{}{}{Hanoi}{}{}
The \emph{Towers of Hanoi} puzzle is a game creaded in the late 1800s by
mathematician \'{E}douard Lucas. The game consists of three pegs, with a set
of disks on one peg, arranged from largest disk on the bottom to smallest disk
at the top.\parend

The goal of the game is to move all of the disks from one peg to another,
following two rules:
\begin{itemize}
  \item You may only move the top disk from one peg to another
  \item You may never place a larger disk on top of a smaller disk
\end{itemize}

There is a simple recursive algorithm for solving the puzzle, given in
Algorithm~\ref{alg:rhanoi}.

\begin{algorithm}[H]
  \vskip6pt
%//  \begin{labeling}{Postconditions}
%//    \item [\bf Preconditions] Source peg $s$ has at least $n$ disks, temporary
%//    peg $t$ and destination peg $d$ have zero or more disks
%//    \item [\bf Postconditions] The top $n$ disks on peg $s$ have been moved to
%//    disk $d$; no other disks are moved
%//  \end{labeling}

  \caption{Recursive solution for Towers of Hanoi}\label{alg:rhanoi}
  \begin{algorithmic}[1]
    \Require{Source peg $s$ has at least $n$ disks, temporary peg $t$ and
    destination peg $d$ have zero or more disks}
  \Ensure{The top $n$ disks on peg $s$ have been moved to disk $d$; no other
    disks are moved}
    \Statex
    \Procedure{RecursiveHanoi}{$n,s,d,t$}
      \If{$n>0$}
        \State {\sc RecursiveHanoi}$(n-1,s,t,d)$
        \Comment\textcolor{blue}{Move $n-1$ disks to temporary peg}
        \State Move top disk from peg $s$ to peg $d$
        \Comment\textcolor{blue}{Move bottom disk in stack of $n$}
        \State {\sc RecursiveHanoi}$(n-1,t,d,s)$
        \Comment\textcolor{blue}{Move $n-1$ disks from temporary peg}
      \EndIf
    \EndProcedure
  \end{algorithmic}
\end{algorithm}
There is also an interesting non-recursive solution as well. Given $n$ disks,
let the smallest disk be designated $d_0$ and the largest disk be $d_{n-1}$.
Label the pegs $p_0$, $p_1$ and $p_2$. Every time the largest disk moves (which
is once), it moves from peg $p_i$ to peg $p_{(i-1)\bmod 3}$. The second-largest
disk, when it moves, always moves from peg $p_i$ to peg $p_{(i+1)\bmod 3}$.
Each disk, when it moves, always moves in the opposite direction to the next
larger disk.\parend
Selecting the disk to move at each step is based on counting in binary. Count
from $1$ to $2^n-1$. At step $j$ of the counting, let $j=(b_{n-1}n_{n-2}\cdots
b_1b_0)_2$ and let $k$ be the smallest integer with $b_k=1$. Move disk $k$ in
the direction it is always supposed to move.\parend

The algorithm for this is given in Algorithm~\ref{alg:ihanoi}.

\begin{algorithm}[H]
  \vskip6pt
  %\begin{labeling}{Postconditions}
  %  \item [\bf Preconditions] Source peg $p_0$ has $n$ disks, temporary
  %  peg $p_1$ and destination peg $p_2$ have zero disks
  %  \item [\bf Postconditions] Source peg $p_0$ and temporary disk $p_1$ have
  %  zero disks, destination peg $p_2$ has $n$ disks
  %\end{labeling}

  \caption{Iterative solution for Towers of Hanoi}\label{alg:ihanoi}
  \begin{algorithmic}[1]
    \Require{Source peg $p_0$ has $n$ disks, temporary peg $p_1$ and destination
      peg $p_2$ have zero disks}
  \Ensure{Source peg $p_0$ and temporary disk $p_1$ have zero disks, destination
    peg $p_2$ has $n$ disks}
    \Statex
    \Procedure{IterativeHanoi}{$n$}
      \State $dir[n-1]\gets-1$
      \Comment\textcolor{blue}{$dir[j]$ is the direction disk $j$ moves}
      \State $loc[n-1]\gets0$
      \Comment\textcolor{blue}{$loc[j]$ is the peg number holding disk $j$}
      \For{$i\gets n-2{\bf\ downto\ }0$}
        \State $dir[i]\gets-dir[i+1]$
        \State $loc[i]\gets0$
      \EndFor
      \Statex
      \For{$i\gets1{\bf\ to\ }2^n-1$}
        \State Let $i=(b_{n-1}b_{n-2}\cdots b_1b_0)_2$
        \Comment\textcolor{blue}{Binary representation of counter $i$}
        \State Let $j$ be the smallest integer with $b_j=1$
        \Statex
        \State $start\gets loc[j]$
        \State $end\gets (loc[j]+dir[j])\bmod3$
        \State $loc[j]\gets end$
        \Statex
        \State Move disk $j$ from $peg_{start}$ to $peg_{end}$
      \EndFor
    \EndProcedure
  \end{algorithmic}
\end{algorithm}
\newpage
Prove that {\sc IterativeHanoi} is correct.\parend
\emph{Hints}\/:
\begin{itemize}
  \item Use weak induction on the number of disks.
  \item What happens if you run this same algorithm, but switch the direction
    each disk moves?
\end{itemize}
\ifanswers
\textcolor{blue}{
\textbf{Answer:}\\[6pt]
Answer goes here
}
\newpage
\fi
