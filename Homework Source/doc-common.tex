% common preamble

%
% this is deprecated, may be removed
%
%\ifshadedsectiontitles
%\usepackage[explicit]{titlesec}
%\fi

% colors
\usepackage{color}
% a better margin than the default article
\usepackage[margin=1in]{geometry}
% for printing listings
\usepackage{listings}
% for algorithms
\usepackage{algorithm}
\usepackage{algpseudocode}
% gotta have tikz
\usepackage{tikz}

\usetikzlibrary{shapes}
\usetikzlibrary{arrows}
\usetikzlibrary{calc,positioning}

%\usepackage{scrextend}
\usepackage{fancyhdr}
\usepackage{lastpage}

\usepackage{arev,tgadventor,tgcursor}
%\usepackage{tex-gyre-math,tgadventor,tgcursor}
\usepackage[semibold]{sourcecodepro}
% use TeX Gyre Adventor as the default font
\renewcommand{\rmdefault}{qag}
% use TeX Gyre Cursor as the tt font
%\renewcommand*{\ttdefault}{qcr}
% sans serif is the base font
\renewcommand*\familydefault{\sfdefault}

% accented letters use a single glyph
\usepackage[T1]{fontenc}
% AMS math symbols and formatting
\usepackage{amssymb,amsmath}
\usepackage{amsthm}

% formatting for problem numbers
\makeatletter
\newcommand{\problemnumber}[5]{\ifx\@empty#3\@empty%
\ifx\@empty#2\@empty%
\,[#1]\\%
\else
\,[#1#4#2]\\%
\fi\else%
\,[#1#4#2#5#3]\\%
\fi%
}
\makeatother

% bold tt font
\newcommand\ttbf[1]{\texttt{\textbf{\large #1}}}

%
% deprecated
%
% macro for section titles. creates a gray box across the page and displays
% the section title in Large size with emphasis. Skips 12pt before and 3pt after
%\newcommand{\sectiontitle}[1]{\vskip12pt\par\noindent\colorbox{gray!40}
%{\parbox{\dimexpr\textwidth\relax}{\Large \emph{#1}}}\vskip3pt\par
%\noindent}

% fancier section titles, shaded for contrast
\ifshadedsectiontitles

%
% this is the old method for setting up shaded section titles, using the
% titlesec package. The problem with this is that at this time it does not
% support accessible headings.
%
%\titleformat{\section}[block]{\normalfont\Large}{}{0pt}{\colorbox{gray!40}
%{\parbox{\dimexpr\textwidth\relax}{\emph{#1}}}}[\vskip-6pt]

%
% shaded section box kludge, draw the box, use kerning to go back to left edge
% and then draw the text. Phantom text is to get vertical box size.
%
\newcommand\sectionbox[1]{\colorbox{gray!40}{\parbox{\dimexpr\textwidth\relax}{%
\phantom{ABCDEFGHIJKLMNOPQRSTUVWXYZabcdefghijklmnopqrstuvwxyz}}}\kern-\textwidth #1}

%
% reset the definition of \section to set size and shape of text and invoke
% \sectionbox at the end.
%
% For all of this to work with accessibility features, the section title must
% be the last tokens generated.
%
\newcounter{chapter}
\newcounter{altchapter}
\makeatletter
\newcommand\chapter{
                      \secdef\@chapter\@schapter
                   }
\def\@chapter[#1]#2{
                      \normalsize \bf #2
                    }
\iflab
\renewcommand\chapter{\@startsection {chapter}{0}{\z@}%
									{0pt}%{-3.5ex \@plus -1ex \@minus -.2ex}%
									{0pt}%{2.3ex \@plus.2ex}%
                                   	{\normalfont\Large\@gobble}}
\else
\renewcommand\chapter{\@startsection {chapter}{0}{\z@}%
									{0pt}%{-3.5ex \@plus -1ex \@minus -.2ex}%
									{0pt}%{2.3ex \@plus.2ex}%
                                   	{\normalfont\Huge\@gobble}}
\fi
   									\renewcommand\section{\@startsection {section}{1}{\z@}%
                                   {-3.5ex \@plus -1ex \@minus -.2ex}%
                                   {2.3ex \@plus.2ex}%
                                   {\normalfont\Large\itshape\sectionbox}}
\renewcommand\subsection{\@startsection {subsection}{2}{\z@}%
                                   {-3.3ex \@plus -1ex \@minus -.2ex}%
                                   {2.1ex \@plus.2ex}%
                                   {\normalfont\Large$\triangleright\quad$\itshape}}
\makeatother

%
% the last part of the kludge adds some negative vertical spacing to bring the
% following text a bit closer to the section box
%
% also centers chapter titles
%
\iflab
\relax
\else
\let\oldchapter\chapter
\renewcommand{\chapter}[1]{\oldchapter{\centerline{#1}}\phantom{X}\\}
\fi
\let\oldsection\section
\renewcommand{\section}[1]{\oldsection{#1}\vskip-6pt}
\let\oldsubsection\subsection
\renewcommand{\subsection}[1]{\oldsubsection{#1}\vskip-6pt}

\fi

\newlength\tindent
\setlength{\tindent}{\parindent}
\setlength{\parindent}{0pt}
\renewcommand{\indent}{\hspace*{\tindent}}

\newcommand\parend{\par\vskip12pt\noindent}

% listings settings
\lstset{
language=C++,
basicstyle=\ttfamily,
    frame=tb, % draw a frame at the top and bottom of the code block
numberstyle=\footnotesize\color[rgb]{0.205, 0.142, 0.73},
commentstyle=\color[rgb]{0.026,0.4,0.095},
        stringstyle=\color[rgb]{0.627,0.126,0.941},
  tabsize=4, % tab space width
  %  showstringspaces=false, % don't mark spaces in strings
    numbers=left, % display line numbers on the left
 %   commentstyle=\color{green}, % comment color
    keywordstyle=\color{blue}, % keyword color
 %   stringstyle=\color{red} % string color
%mathescape=true,
}

%
% this kludge comes from Ulrike Fischer from the tagged LaTeX project.
%
% it defines a taglstlisting environment
%

%\makeatletter
%\ExplSyntaxOn
%\DeclareInstance{blockenv}{lstlisting}{display}
%{
%  env-name       = lstlisting,
%  tag-name       = verbatim,
%  tag-class      = ,
%  tagging-recipe = standard,
%  inner-level-counter  = ,
%  level-increase = false,
%  setup-code     = ,
%  block-instance = displayblock ,
%  inner-instance = ,
%  final-code     =  \tl_set:Nn \l__tag_para_main_tag_tl {codeline}\tagtool{paratag=Code},
%}
%\ExplSyntaxOff
%\lstnewenvironment{taglstlisting}[2][]{%
%     \UseInstance{blockenv}{lstlisting} {}
%     \lst@TestEOLChar{#2}%
%     \lstset{#1}%
%     \csname\@lst @SetFirstNumber\endcsname%
%   }{%
%     \@nobreakfalse
%     \csname\@lst @SaveFirstNumber\endcsname%
%     \endblockenv
%   }
%\makeatother

%
% these may be deprecated, I don't think I use any of them
%

% common big Theta values
%\newcommand\consttime{$\Theta\left(1\right)$}
%\newcommand\lgntime{$\Theta\left(\lg n\right)$}
%\newcommand\ntime{$\Theta\left(n\right)$}
%\newcommand\nlgntime{$\Theta\left(n\lg n\right)$}
%\newcommand\nsquaredtime{$\Theta\left(n^2\right)$}

%\newcommand\titlenoduedate{%
%\thispagestyle{empty}%
%\cfoot{\thepage\ of \pageref{LastPage}}%
%\lhead{\coursetitle}%
%\chead{\shorttitle}%
%\rhead{\shortcourseterm}%
%\noindent{\Huge\coursetitle\\[3pt]\catalognum\\[3pt]}%
%\emph{\large\courseterm}\\[3pt]%
%}

%\newcommand\titlewithduedate{%
%\thispagestyle{empty}%
%\cfoot{\thepage\ of \pageref{LastPage}}%
%\lhead{\coursetitle}%
%\chead{\shorttitle}%
%\rhead{\shortcourseterm}%
%\noindent{\Huge\coursetitle\\[3pt]\catalognum\\[3pt]}%
%\emph{\large\courseterm}\\[3pt]%
%\noindent\makebox[\linewidth]{\rule{\linewidth}{0.4pt}}\\
%{\large\doctitle}\\
%\emph{\duedate}\\
%}

\pagestyle{fancy}

% for hypersetup
\usepackage{hyperref}
