\documentclass{article}

% this sets the document margins to 1 inch, much better than the default
\usepackage[margin=1in]{geometry}

% bring in AMS symbols and environments
\usepackage{amssymb,amsmath}
\usepackage{amsthm}

\begin{document}

There are two types of lists:
% The itemize environment generates unnumbered lists
\begin{itemize}
  % each item is a bullet point
  \item Unnumbered lists
  \item Numbered lists
\end{itemize}

Unnumbered lists are straightforward. They are created with the {\tt itemize}
environment. Individual bulleted items are created with the {\tt \textbackslash
item} command.\\

You can also create sublists, as in the following example:
\begin{itemize}
  \item First item
  \item Second item
  % For a sublist, you just embed an itemize environment inside the current
  % environment
  \begin{itemize}
    \item \LaTeX{} automatically adjusts the bullets for each level of
    indentation
    \item Note that if an item is really long, as this item is, that each line
    of text is properly indented. This is an extra sentence to make sure this
    item is really, really long and takes up more than one line on the page.
  \end{itemize}
  \item Third item
\end{itemize}

You can embed up to four levels of lists. It is rare to need more than two
levels.\\

Numbered lists are created with the {\tt enumerate} environment. Here's an
example:
\begin{enumerate}
  \item First step
  \item Second step
  \begin{enumerate}
    \item Sublists work here too
    \item See how numbering has changed to lettering here
  \end{enumerate}
  \item Numbering resumes when a sublist finishes
  \begin{itemize}
    \item An example of embedding an unnumbered sublist inside a numbered list
    \item You can mix and match as needed
  \end{itemize}
\end{enumerate}

That's all there is to basic lists. There are many ways to customize a list; I
customize the second level when writing homework problems with parts,
eliminating the parentheses around the letter.\\

A good reference for creating and customizing lists is at
\begin{center}
  {\tt\footnotesize https://www.overleaf.com/learn/latex/Lists\#The\_enumerate\_environment\_for\_numbered\_.28ordered.29\_lists}
\end{center}
\newpage

\LaTeX{} has the ability to create tables. Although they can be a bit difficult
to work with if you want to do anything fancy, creating a basic table is fairly
simple.\\

To create a table, use the {\tt tabular} environment. When you begin the
environment, you must also specify the number of columns and the justification
of each column --- left, right or centered. For example, the command
\begin{center}
  {\tt \textbackslash begin\{tabular\}\{cclrr\}}
\end{center}
creates a table with five columns --- the first two columns are centered, the
third column is left-justified and the last two columns are right-justified.\\

Content inside the {\tt tabular} environment is divided into rows. A row is all
content up to an end-of-line marker {\tt \textbackslash\textbackslash}. Within
a row, columns are separated by an ampersand ({\tt \&}) character. Note that if
there are $n$ columns, there must always be at most $n-1$ ampersands on a row.\\

Column widths are adjusted to hold the widest content in the column, plus some
padding. Here's an example illustrating this:\\[6pt]%add 1/2 line of space

% create table with two columns, center and right-justified
\begin{tabular}{cr}
  {\bf Word} & {\bf Count}\\% \bf is boldface
  cat & $3$\\% note: good practice to put numbers in math mode
  horse & $5$\\
  octopus & $7$\\
  antidisestablishmentarianism & $28$\\
\end{tabular}
\vskip12pt % needed to add some space between end of table and start of text

Note that there are no borders, and the table is not centered on the page. To
center the table, place the {\tt tabular} environment inside a {\tt center}
environment (see next example).\\

Horizontal borders are always above the content line. To add a horizontal
border, use the {\tt \textbackslash hline} command before the content of the
line. This will place the border above the content. To get a border below the
last line, add a line at the bottom with the {\tt \textbackslash hline} command
and no content. You can use two consecutive {\tt \textbackslash hline} commands
to get a double line.\\

To create vertical borders, insert vertical bars ({\tt |}) in the column
specification wherever you want a vertical border. This includes before the
first column and after the last column. For example, to add vertical borders to
the previous example, you can use {\tt \{|c|r|\}} as the column specification.\\

Here is an example using borders; note the use of a double line to help separate
the column headings from the rest of the table.\\

% math mode symbols:
% \lnot - logical not
% \to - right arrow
% \iff - if and only if (biconditional)
\begin{center} % center the table
  \begin{tabular}{|c|c|c|c|c|c|c|} % 7 columns, all centered, vertical borders
    \hline % starting with a top border
    $p$ & $q$ & $\lnot p$ & $\lnot q$ & $p\to q$ & $\lnot q\to \lnot p$ &
    $(p\to q)\iff(\lnot q\to\lnot p)$\\ \hline\hline % separate headings from
    % the rest of the table
    T & T & F & F & T & T & T\\ \hline % top border of next line
    T & F & F & T & F & F & T\\ \hline
    F & T & T & F & T & T & T\\ \hline
    F & F & T & T & T & T & T\\ \hline % border at the bottom of the table
  \end{tabular}
\end{center}
\vskip12pt

Final thought: Instead of explaining the symbols and commands here, I've added
comments in the \LaTeX{} document to explain them. Have a look at the source
file; comments begin with {\tt \%} and go to the end of the line.
\end{document}
