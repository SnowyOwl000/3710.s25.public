\documentclass{article}

\begin{document}

This is a basic LaTeX (or \LaTeX{} if you want to be fancy) document.\\[12pt]

Documents have two main sections: a preamble and a body.\\[12pt]

The preamble always starts with the {\tt \textbackslash documentclass} command.
This indicates the type of document you are creating. Although there are several
document types, you're most likely to want to use the {\tt article} class, which
this document uses.

The preamble also contains definitions of commands and symbols, as well as
imports of other \LaTeX{} packages. There are hundreds of imports available in a
complete \LaTeX{} installation. Here are some that may prove useful for you:
\begin{itemize}
  \item {\tt \textbackslash usepackage\{geometry\}} --- allows you to change
  the document margins, among other things (note the margins on this document
  are quite wide; that's the default setting)
  \item {\tt \textbackslash usepackage\{amsmath\}} --- adds more mathematical
  typesetting capabilities
  \item {\tt \textbackslash usepackage\{amssymb\}} --- adds more mathematical
  symbols
  \item {\tt \textbackslash usepackage\{amsthm\}} --- adds an environment for
  typesetting theorems, axioms, \emph{etc}
  \item {\tt \textbackslash usepackage\{algorithm\}}\\
    {\tt \textbackslash usepackage\{algpseudocode\}} --- adds an environment for
    writing algorithms in pseudocode
  \item {\tt \textbackslash usepackage\{listings\}} --- adds an environment for
    formatting program listings
  \item {\tt \textbackslash usepackage\{tikz\}} --- adds TikZ commands. TikZ is
  a very complex subsystem which is useful for creating vector graphics (for
  example, trees and graphs). I'll provide more information in a separate
  document.
\end{itemize}
\vskip 12pt
The document body consists of everything from the {\tt \textbackslash
begin\{document\}} command down to the {\tt \textbackslash end\{document\}}
command, which should be the last line of the file. Everything in this
\emph{environment} is rendered in the output document.\newpage

The body is essentially a sequence of blocks of symbols. These blocks can be:
\begin{itemize}
  \item Paragraphs consisting of text and symbols
  \item Blocks of mathematics text and symbols (such as a proof)
  \item Itemized or enumerated lists
  \item Code listings or algorithms
  \item Images (either drawn or imported)
\end{itemize}
There are other types of blocks as well, but you're most likely to use these
five.\\[12pt]

When you begin typing in the {\tt document} environment, by default you're
creating a paragraph. The paragraph continues until you start a new paragraph
or begin an environment for one of the other blocks listed above. To start a new
paragraph, simply add a blank line to your input document.\\[12pt]

The other blocks are environments, which are delimited by {\tt \textbackslash
begin\{\ldots\}} and {\tt \textbackslash end\{\ldots\}} commands. Aside from
the {\tt document} environment, this document also uses the \emph{itemize}
environment, which creates bulleted lists.\\[12pt]

The other commands used in this document are explained here:
\begin{itemize}
  \item {\tt \textbackslash LaTeX} --- produces the fancy \LaTeX{} logo
  \item {\tt \{\}} --- this doesn't produce any output, but it has two uses:
    \begin{itemize}
      \item Ending a command without messing up spacing. Try, for example,
      removing the {\tt \{\}} after {\tt \textbackslash LaTeX} and render the
      document. Notice the change in spacing.
      \item Preventing two adjacent symbols from combining. Try changing
      {\tt -\{\}-} to {\tt -{}-} in the dashes explanation and render the
      document. Notice that instead of two hypens, it is rendered as an en-dash.
    \end{itemize}
  \item {\tt \textbackslash\textbackslash} --- line break. The additional
  {\tt [12pt]} adds an extra $12$ points of vertical space before the next
  line. There are $72.27$ points in an inch.
  \item {\tt \textbackslash tt} --- switches the font to monospace /
  ``typewriter'' font.\\ Important note: switching fonts (\emph{typefaces}) is a
  ``set it and forget it'' action; the switch continues until it is switched
  again. Place the command and the text that you want to appear in monospace
  in curly braces {\tt \{\ldots\}}.
  \item {\tt \textbackslash textbackslash} --- displays a backslash character
  {\tt \textbackslash}
  \item {\tt \textbackslash item} --- creates a new bullet in an itemized list
  \item {\tt \textbackslash \{} and {\tt \textbackslash \}} --- displays open
  and close curly braces
  \item {\tt -{}-{}-} --- this is an \emph{em-dash}, used for separating parts
  of a sentence. There is also an \emph{en-dash} {\tt -{}-} which is used for
  the dash between the start and end of a range of values (like 1--10), and a
  \emph{hyphen} which is used between individual words, or to split a word
  between lines (which \LaTeX{} does automatically, as in ``creating''
  on the previous page).
  \item {\tt \textbackslash emph} --- italicizes (\emph{emphasizes}) everything
  in the curly braces that follow
  \item {\tt \textbackslash vskip} --- adds vertical space after the end of an
  environment
  \item {\tt \$} --- starts and ends \emph{math mode}. Math mode is used to
  render numbers, math symbols and formulas. Note that {\tt \$} \emph{always}
  appear in pairs. More about math mode in other documents.
  \item {\tt \textbackslash ldots} --- this produces an ellipsis
  \ldots
  \item {\tt \textbackslash "\{\ldots\}} --- draws an umlaut (or
  \emph{diaresis}) over a vowel
  \item {\tt \textbackslash TeX} --- typesets the fancy \TeX{} logo
\end{itemize}
\vskip12pt

Finally, if you're \emph{really} interested in learning \LaTeX{}, here are a few
references that I use:
\begin{itemize}
  \item \emph{Guide to \LaTeX{}}, fourth edition, by Kopka and Daly
  \item \emph{The \LaTeX{} Companion}, second edition, by Mittelbach and Goosens
  \item \emph{The \LaTeX{} Graphics Companion}, second edition, by Goosens
  \emph{et al}.
  \item \emph{Math Into \LaTeX{}}, by Gr\"{a}tzer
  \item \emph{The \TeX book}, by Knuth
\end{itemize}
Plus, there are many online references for \LaTeX{} information.

\end{document}
